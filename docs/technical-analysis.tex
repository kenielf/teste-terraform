\documentclass{article}

\usepackage{fancyvrb}
\VerbatimFootnotes
\usepackage{minted}
\usepackage{indentfirst}
\usepackage{hyperref}
\hypersetup{
    colorlinks=true,
    linkcolor=black,
    filecolor=magenta,
    urlcolor=cyan,
    pdftitle={NeoTasker - Documentação},
    pdfpagemode=FullScreen,
}
\urlstyle{same}


\begin{document}
\tableofcontents
\pagebreak

\section{Configuração do Terraform para AWS}
Terraform é uma ferramenta de \textbf{infraestrutura como código \textit{IaC}} para orquestrar diferentes componentes de infraestrutura, criando e destruindo recursos, e efetuando modificações quando necessário. Sua configuração é realizada a partir de arquivos com a extensão \Verb{tf}.

Para o escopo definido no arquivo \Verb{main.tf} contido neste repositório, serão criados recursos para a \textbf{Amazon Web Services \textit{(AWS)}} para subir em nuvem uma infraestrutura com recursos de rede e segurança complementando o recurso de \textit{compute}.

\subsection{Provedor}
Localizado ao início do arquivo, a primeira seção define qual é o provedor -- um plugin utilizado para gerenciar os recursos que são declarados na configuração.
\begin{minted}[autogobble]{terraform}
    provider "aws" {
      region = "us-east-1"
    }
\end{minted}

O parâmetro \Verb{region} define em qual região serão criados os recursos, neste caso \Verb{us-east-1} se refere a região de \textbf{Virgínia do Norte, Estados Unidos}.

\subsection{Variáveis}
Logo abaixo, são definidos variáveis que serão reutilizadas em diferentes trechos da configuração.
\begin{minted}[autogobble]{terraform}
    variable "projeto" {
      description = "Nome do projeto"
      type        = string
      default     = "VExpenses"
    }

    variable "candidato" {
      description = "Nome do candidato"
      type        = string
      default     = "ChrystianFranklin"
    }
\end{minted}

Ambas essas variáveis contém campos de descrição, tipo e valor padrão -- que podem ser sobrescritos por variáveis de ambiente, esses valores podem ser sobrescritos em diferentes módulos, ou por argumentos ao executável\footnote{
    \Verb{terraform apply -var="candidato=OutroCandidato"}
}, ou por variáveis de ambiente\footnote{
    \Verb{TF_VAR_candidato="OutroCandidato" terraform apply}
}, ou até mesmo por um \href{https://developer.hashicorp.com/terraform/language/values/variables\#variable-definitions-tfvars-files}{arquivo dedicado de variáveis}.

Apesar de não estar incluso nesse trecho, as variáveis podem conter configuração de validação a partir de condições, podendo verificar diferentes aspectos, como por exemplo o tamanho mínimo e máximo de uma variável e prefixação, e quando não são atendidas, uma mensagem de erro pode ser exibida a partir do parâmetro de \Verb{error_message}.

\subsection{Recursos}
\subsubsection{Par de Chaves TLS}
Nessa seção são criadas dois recursos principais, uma chave privada de TLS e um par de chaves da AWS (a partir da chave TLS), que serão usadas para acessar a instância EC2 que será criada.
\begin{minted}[autogobble]{terraform}
    resource "tls_private_key" "ec2_key" {
      algorithm = "RSA"
      rsa_bits  = 2048
    }
\end{minted}

O algorítmo utilizado para a criação da chave é o RSA com tamanho de 2048 bits, considerado seguro\footnote{
    Uma implementação segura de RSA requer \textit{blinding} para evitar \href{https://www.cs.sjsu.edu/faculty/stamp/students/article.html}{ataques de timing}.
} e utilizado por várias plataformas como tipo de chave principal.

\begin{minted}[autogobble]{terraform}
    resource "aws_key_pair" "ec2_key_pair" {
      key_name   = "${var.projeto}-${var.candidato}-key"
      public_key = tls_private_key.ec2_key.public_key_openssh
    }
\end{minted}

Esse trecho cria o par de chave na AWS utilizando as variáveis de \textbf{projeto} e \textbf{candidato}, a partir da chave pública gerada no trecho anterior, com o formato do OpenSSH.

\subsubsection{Virtual Private Cloud (VPC)}
\subsubsection{Subnet}
\subsubsection{Gateway}
\subsubsection{Route Table \& Associations}
\subsubsection{Security Group (SSH)}
\subsubsection{Amazon Machine Image (AMI)}
\subsubsection{Elastic Compute Cloud (EC2)}
\subsection{Outputs}
\subsubsection{Chave de Acesso Privada}
\subsubsection{Endereço IP Público}


\end{document}
